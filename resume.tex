\documentclass{resume}

\renewcommand{\categoryfont}{\sc}

%
% set the space used for category titles here:
% use the same value for oddsidemargin and marginparwidth [the latter 
% 		will be reset to account for marginparsep]
% 
\setlength{\oddsidemargin}{1in}
\setlength{\marginparwidth}{1in}
% 
% calculate other dimensions [textwidth and evensidemargin] 
% in function of oddsidemargin and marginparwidth: 
% would be nicer to put in the class file...
%
\addtolength{\marginparwidth}{-\marginparsep}
 \setlength{\evensidemargin}{\oddsidemargin}
 \setlength{\textwidth}{\paperwidth}
 \addtolength{\textwidth}{-2in}
 \addtolength{\textwidth}{-2\oddsidemargin}
 \addtolength{\textwidth}{\marginparwidth}
 \addtolength{\textwidth}{\marginparsep}
%
%
\setlength{\topmargin}{-0.5in}
%
%
\renewcommand{\labelcitem}{$\diamond$}
\renewcommand{\labelitemi}{$\cdot$}
\newcommand{\first}{$1^{\mbox{\scriptsize st}}$\ }
\newcommand{\second}{$2^{\mbox{\scriptsize nd}}$\ }
\newcommand{\third}{$3^{\mbox{\scriptsize rd}}$\ }

\author{Amin Khorsandiaghai}
% ------ Address --------------------------------------------------------

\address{Blomgatan 20\\
	 16960 Solna\\
	 Sweden\\}
        {+46725074461 (mobile)\\
	\mbox{\small\tt aminkhorsandi@gmail.com}\\
	\mbox{\small\tt amin.khorsandiaghai@ericsson.com}}

\begin{document}
\maketitle
% ------- Education ---------------------------------------------------

\begin{category}{Education}
\citem{Uppsala University}, Uppsala, Sweden. \\
Diploma [M.Sc.] in computer science, (grades: 87\%).\\
Thesis title: {\em Implementing typed Psi-Calculi.}\\
Key courses: Large scale programming, Verification methods, Data mining, 
From domains to requirements, Test methodology,
Software architecture with Java, Parallel programming, Compiler design project. 
\citem{Tabriz University}, Tabriz, Iran.\\
Diploma [B.Sc.] in computer science (software engineering option), May 2004.\\
Key courses: Information system analysis and design, Computer networks, 
Management information systems, Algorithm design and analysis, Database technology.
\end{category}

\begin{center}
\line(1,0){400}
\end{center}

% --------- Research ----------------------------------------------------

\begin{category}{Open Source Projects}

\citembullet
Working in a XFT (4 developers) to implement a ray-tracer program based on 
existing specification in Java-SE.
The aim of the project was to write a high quality program which is easy 
to read and understand, well-structured, well-documented, portable and easily maintainable.   
The project includes writing unit tests using Junit and test automation 
using Ant/Maven to ensure the functionality of the program.
Profiling also used to analyze the resource consumption of program and to optimize it. 
We use Eclipse IDE for this project.
Svn was the project's version control system (nowadays we maintain the code in Git):
https://github.com/coffeMug/ray-tracer.git

\citembullet A e-commerce project (on-line PC shop) based on MVC design 
pattern (n-tier architecture) and implemented in Java-EE development platform 
using Netbeans.

\citembullet
Implementing the time-stamp based concurrency control protocol for a 
distributed client-server system written in Erlang/OTP.

\citembullet
During the course ``from domain to requirements'' I was a member of a team 
working on a software engineering project to investigate the domain of a 
water distribution system. 
Later a formal prescription of requirements for the system obtained 
and was written in RAISE specification language

\citembullet
Writing formal specification of a double deck elevator controller 
software in B Specification Language using AtelierB tool 
(later C code produced as the final refinement step).

\citembullet Writing a compiler for a small language, uC, and generating assembly 
code for the MIPS architecture.
The project was in a sequence of steps, where each step was a separately specified 
assignment to complete a specific component of the compiler.
The generated assembly code was executed by the SPIM MIPS simulator. 
The compiler was written using Standard ML programming language:
https://github.com/coffeMug/ML-MuC-compiler.git 

\end{category}

\begin{center}
\line(1,0){400}
\end{center}

%\newpage
% -------- Work experience --------------------------------------------

\begin{category}{Work \\experiences}

\citem{Software Developer}, Ericsson , Stockholm, Sweden (November 2012 - Now)\\ \\
Working for SBG (Session Border Gateway) node in CBA migration project which is a project 
aimed for transferring the existing legacy code base (implemented in Erlang/OTP) into the 
new architecture (Component Based Architecture deployed on top of Ericsson's cloud infrastructure). 
Our development strategy is based on TDD and XP. We also develop and extend unit tests using Eunit framework, 
function tests to examine the different functionality of the SGC (Session Gateway Controller) application 
based on Erlang common test framework and we also have system tests to test the overall system in a real 
world environment (heavy load, live traffic etc.). 
We use Jenkins as our CI (Continuous Integration) tool in order to improve our software 
quality and reduce release and delivery time. I am member of an Agile team and our team-work is 
organized based on Scrum-ban method.

\citem{Research Assistant}, Mobility Research Group, Department of Information Technology, 
Uppsala University, Uppsala, Sweden (July 2012 - September 2012)\\ 
\begin{itemize}
 \item Studying the theory of type systems and mobile processes with a 
  focus on the typed Psi-calculi theory and its implementation issues.
 \item Implementing the type system for different instances of Psi-calculi 
  using the Standard ML programming language.
\end{itemize}

\citem{Software Developer}, Tehran Water Company, Tehran, Iran (May 2005 - August 2009)\\ 
\begin{itemize}
 \item Developing and maintaining the tele-control system software (on the control center site based 
  on \texttt{C++} as well as remote stations written in \texttt{C\#}).
 \item Maintenance of RTU (remote transmission unit) devices and measurement instruments software.
 \item Converting the old documentation system of the company into a upscale 
  database management system (Oracle).
\end{itemize}

\citem{Front-end and Back-end Developer (part-time)}, SCWMRI Research Institute, 
Tehran, Iran (April 2007 - September 2008)
\begin{itemize} 
\item Design and implementation of a CMS (content management system) as a 
 digital portal for institute.
\item Designing of a database system for data manipulation of the CMS using MS-Sql.
\end{itemize}

\end{category}

\begin{center}
\line(1,0){400}
\end{center}
% ------- Skills ------------------------------------------------------
\begin{category}{Skills}
\citem {Programming languages and skills:}\hspace{1 mm}OOP (Java, \texttt{C++}), FP (Erlang, StandardML), 
shell scripting (bash and tcsh), C and Groovy.

\citem {Languages:}\hspace{1 mm}Persian as mother tongue, fluent spoken/written English,
Sfi level D Swedish.

\citem {Agile:}\hspace{1 mm}practice Scrum methodology in a XFT during 
the ``large scale programming'' course. Used Kanban during my master thesis project. 
Have Worked in Scrum-ban teams at the company of Ericsson.

\citem{Version control systems:}\hspace{1 mm}Subversion, Git, Clearcase. 

\citem{Automated testing and verification:}\hspace{1 mm}in-depth knowledge of unit 
testing with JUnit/EUnit, function testing (Erlang common-test framework) 
and integration and system testing.

\citem {Others:}\hspace{1 mm}extensive understanding of large scale 
Software engineering in all its aspects from requirements gathering 
to the design and development; testing and delivery to the customer site. 
Good knowledge of Cloud and Virtualization technologies, Continues Integration and delivery, 
Linux (workstation/cluster), Object Oriented design and functional programming.
Good in troubleshooting and solving problems.

\end{category}

\begin{center}
\line(1,0){400}
\end{center}

% -------- Reference --------------------------------------------

\begin{category}{Reference} 
\citemnobullet Available on request.
\end{category}

\end{document}
